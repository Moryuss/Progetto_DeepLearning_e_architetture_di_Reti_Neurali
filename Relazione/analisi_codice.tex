\documentclass[a4paper, 11pt]{article}
\usepackage[utf8]{inputenc}
\usepackage[T1]{fontenc}
\usepackage{lmodern}
\usepackage[italian]{babel}
\usepackage{geometry}
\usepackage{hyperref}
\usepackage{fancyhdr}
\usepackage{graphicx}

\geometry{a4paper, top=2.5cm, bottom=2.5cm, left=2.5cm, right=2.5cm}
\hypersetup{
    colorlinks=true,
    linkcolor=blue,
    filecolor=magenta,      
    urlcolor=cyan,
    pdftitle={Analisi Codice Face Recognition},
    pdfpagemode=FullScreen,
}

\pagestyle{fancy}
\fancyhf{}
\fancyhead[L]{Analisi Codice Sorgente}
\fancyhead[R]{Progetto di Riconoscimento Facciale}
\fancyfoot[C]{\thepage}

\title{\textbf{Analisi del Codice Sorgente \ Progetto di Riconoscimento Facciale}}
\author{Ghidini Matteo}
\date{\today}

\begin{document}

\maketitle
\tableofcontents
\newpage

\section{Introduzione}
Questo documento analizza la struttura del codice sorgente del progetto di riconoscimento facciale. L'obiettivo è descrivere lo scopo di ogni file Python nella directory \texttt{src/} e mappare le relazioni e le dipendenze tra di essi per fornire una visione chiara dell'architettura complessiva.

\section{Architettura Generale}
Il sistema è suddiviso in tre aree principali:
\begin{enumerate}
    \item \textbf{Componenti Core}: Moduli riutilizzabili che forniscono funzionalità atomiche come il rilevamento di volti (\texttt{detector.py}) e la generazione di embedding (\texttt{recognizer.py}).
    \item \textbf{Script di Preparazione Dati}: Programmi utilizzati per processare i dati offline. Questo include l'aumento dei dati (\texttt{augment.py}) e l'estrazione degli embedding da salvare su disco (\texttt{image\_to\_embedding.py}, \texttt{extract\_embeddings.py}).
    \item \textbf{Script Eseguibili (Entrypoint)}: I programmi principali che l'utente avvia per eseguire le diverse funzionalità del progetto, come il riconoscimento da webcam (\texttt{main\_camera.py}) o la ricerca di sosia (\texttt{main\_look\_alike\_*.py}).
\end{enumerate}
Il file \texttt{config.py} agisce come un punto di configurazione centrale, fornendo percorsi e costanti a tutti gli altri moduli.

\newpage

\section{Analisi dei File Sorgente}

\subsection{config.py}
\begin{description}
    \item[Scopo:] Centralizza tutte le costanti e i percorsi utilizzati nel progetto. Definisce dove trovare i modelli, i dataset, le immagini da classificare e dove salvare gli output come gli embedding.
    \item[Relazioni:] Viene importato da quasi tutti gli altri file del progetto che necessitano di accedere a risorse del filesystem. Questo approccio disaccoppia la logica del codice dai percorsi specifici.
\end{description}

\subsection{detector.py}
\begin{description}
    \item[Scopo:] Contiene la classe \texttt{FaceDetector}, che utilizza un modello YOLO (caricato tramite la libreria \texttt{ultralytics}) per individuare i volti all'interno di un'immagine.
    \item[Funzionalità:] Fornisce metodi per rilevare i volti restituendo i loro riquadri di delimitazione (bounding box) e, opzionalmente, i punti di riferimento facciali (keypoints).
    \item[Relazioni:] È una dipendenza fondamentale per tutti gli script che devono processare un volto, come \texttt{image\_to\_embedding.py}, \texttt{extract\_embeddings.py} e tutti i file \texttt{main\_*.py}.
\end{description}

\subsection{recognizer.py}
\begin{description}
    \item[Scopo:] Contiene la classe \texttt{FaceRecognizer}. Il suo compito è prendere un'immagine di un volto (già ritagliata) e trasformarla in un vettore numerico ad alta dimensionalità, chiamato "embedding".
    \item[Funzionalità:] Utilizza un modello di deep learning pre-addestrato (basato su architetture come FaceNet/VGGFace2) per generare un embedding normalizzato (L2). Questi embedding possono essere poi confrontati tramite similarità cosenica per determinare se due volti appartengono alla stessa persona.
    \item[Relazioni:] È una dipendenza chiave per \texttt{image\_to\_embedding.py}, \texttt{extract\_embeddings.py} e per tutti gli script \texttt{main\_*.py} che eseguono il riconoscimento o la ricerca di similarità.
\end{description}

\subsection{utils.py}
\begin{description}
    \item[Scopo:] Libreria di funzioni di utilità generiche, utilizzate in più parti del progetto per evitare la duplicazione del codice.
    \item[Funzionalità:]
    \begin{itemize}
        \item Funzioni di preprocessing delle immagini per adattarle agli input dei modelli.
        \item Funzioni per caricare gli embedding salvati su file \texttt{.npz}.
        \item La funzione \texttt{recognize\_faces}, che orchestra il rilevamento e il riconoscimento su un'immagine.
        \item Funzioni per disegnare etichette e riquadri sulle immagini di output.
        \item Funzioni per la ricerca dei top-k volti più simili.
    \end{itemize}
    \item[Relazioni:] Importato da tutti i file \texttt{main\_*.py} e da altri script che necessitano di queste funzionalità ausiliarie.
\end{description}

\subsection{augment.py}
\begin{description}
    \item[Scopo:] Script per l'aumento dei dati (data augmentation). Aumenta il numero di immagini nel dataset applicando trasformazioni casuali (flip, luminosità, crop) alle immagini originali.
    \item[Funzionalità:] Per ogni persona nel dataset, controlla se il numero di immagini agumentate è inferiore a una soglia e, in caso affermativo, ne genera di nuove.
    \item[Relazioni:] È uno script di preparazione dati. Non è una dipendenza di altri moduli, ma il suo output (le immagini aumentate) viene processato da \texttt{image\_to\_embedding.py}.
\end{description}

\subsection{image\_to\_embedding.py}
\begin{description}
    \item[Scopo:] Script cruciale per la preparazione del database di riconoscimento. Scansiona il dataset (immagini originali e aumentate), rileva i volti, calcola i loro embedding e li salva in un file compresso \texttt{embeddings.npz} per ogni persona.
    \item[Flusso:]
    \begin{enumerate}
        \item Itera su ogni persona nel dataset.
        \item Per ogni immagine, usa \texttt{detector.py} per trovare il volto.
        \item Passa il volto ritagliato a \texttt{recognizer.py} per ottenere l'embedding.
        \item Salva tutti gli embedding di una persona in un unico file \texttt{.npz}.
    \end{enumerate}
    \item[Relazioni:] Utilizza \texttt{detector.py} e \texttt{recognizer.py}. Produce i file \texttt{.npz} che sono l'input fondamentale per \texttt{main\_camera.py} e \texttt{main\_image.py}.
\end{description}

\subsection{extract\_embeddings.py}
\begin{description}
    \item[Scopo:] Simile a \texttt{image\_to\_embedding.py}, ma specifico per la funzionalità "look-alike". Processa due cartelle separate (\texttt{people} e \texttt{known\_people}) e crea due database di embedding distinti (\texttt{people.npz} e \texttt{known.npz}).
    \item[Relazioni:] Utilizza \texttt{detector.py} e \texttt{recognizer.py}. I suoi file di output sono usati da \texttt{main\_look\_alike\_offline.py} e \texttt{main\_look\_alike\_online.py}.
\end{description}

\subsection{initializer.py}
\begin{description}
    \item[Scopo:] Contiene un solo metodo, per inizializzare detector e recognizer. Prende due path che portano ai rispettivi pesi e crea i modelli  .
    \item[Relazioni:] Dipende da FaceDetector e FaceRecognizer per creare i modelli e dalla classe InceptionResnetV1 per la struttura del modello.
\end{description}



\subsection{main\_camera.py}
\begin{description}
    \item[Scopo:] Entrypoint principale per il riconoscimento facciale in tempo reale tramite webcam.
    \item[Flusso:]
    \begin{enumerate}
        \item Carica in memoria tutti gli embedding del dataset usando \texttt{utils.load\_dataset\_embeddings}.
        \item In un ciclo infinito, cattura un frame dalla webcam.
        \item Usa la funzione \texttt{utils.recognize\_faces} (che a sua volta usa \texttt{detector} e \texttt{recognizer}) per trovare e identificare i volti nel frame.
        \item Disegna i risultati sul frame e lo mostra a schermo.
    \end{enumerate}
    \item[Relazioni:] Dipende da \texttt{detector.py}, \texttt{recognizer.py} e \texttt{utils.py}.
\end{description}

\subsection{main\_image.py}
\begin{description}
    \item[Scopo:] Entrypoint per il riconoscimento facciale su un set di immagini statiche.
    \item[Flusso:] Simile a \texttt{main\_camera.py}, ma invece di un ciclo sulla webcam, itera sui file di immagine presenti in una cartella specificata in \texttt{config.py}.
    \item[Relazioni:] Dipende da \texttt{detector.py}, \texttt{recognizer.py} e \texttt{utils.py}.
\end{description}

\subsection{main\_look\_alike\_offline.py}
\begin{description}
    \item[Scopo:] Esegue una ricerca "look-alike" (sosia) offline. Confronta ogni persona nel database "known" con tutte le persone nel database "people" per trovare le corrispondenze più simili.
    \item[Flusso:] Carica i due database di embedding pre-calcolati da \texttt{extract\_embeddings.py} e calcola la similarità cosenica per trovare i top-K match per ogni persona "known".
    \item[Relazioni:] Dipende da \texttt{utils.py} per caricare gli embedding e trovare i match. Non necessita di \texttt{detector} o \texttt{recognizer} perché lavora su embedding già esistenti.
\end{description}

\subsection{main\_look\_alike\_online.py}
\begin{description}
    \item[Scopo:] Esegue una ricerca "look-alike" in tempo reale. Rileva un volto dalla webcam e cerca la persona più simile all'interno del database "people".
    \item[Flusso:]
    \begin{enumerate}
        \item Carica il database di embedding "people".
        \item In un ciclo, cattura un frame, rileva un volto con \texttt{detector.py} e calcola il suo embedding al volo con \texttt{recognizer.py}.
        \item Confronta l'embedding appena calcolato con quelli del database per trovare il miglior match.
    \end{enumerate}
    \item[Relazioni:] Dipende da \texttt{detector.py}, \texttt{recognizer.py} e \texttt{utils.py}.
\end{description}

\subsection{camera.py e old\_augment.py}
\begin{description}
    \item[Scopo:]
    \begin{itemize}
        \item \texttt{camera.py}: Fornisce una semplice classe wrapper per \texttt{cv2.VideoCapture}. Attualmente non sembra essere utilizzata da nessuno degli script principali, che chiamano direttamente \texttt{cv2}. Potrebbe essere un residuo o un componente per usi futuri.
        \item \texttt{old\_augment.py}: Sembra una versione precedente e più semplice di \texttt{augment.py}. Probabilmente è stato deprecato in favore della versione più robusta.
    \end{itemize}
\end{description}


\section{Flow di utilizzo applicazione}
\subsection{Utilizzo immediato}
Se si vuole far semplicemente partire l'applicativo basta scrivere su bash shell:\\
\texttt{python -m src.main\_*}
Questo farà partire una dei main presenti in src.
\subsection{Aggiunta di persona al dataset di riconoscimento}
Se invece si vuole aggiungere persone alla knowledge base serve fare alcuni passaggi aggiuntivi:
\begin{enumerate}
    \item Aggiungere alla dir \texttt{data/dataset} la dir con nome quello della persona che si vuole aggiungere alla KB
    \item Eseguire su bash shell: \\
    \texttt{python -m augment} \\
    Questo agumenterà l'immagini presenti nel dataset
    \item Per creare gli embeddings dalle immagini:
    \texttt{python -m src.image\_to\_embedding}
    \item Si può far partire i main\_camera o main\_image con\\
    \texttt{python -m src.main\_*}
\end{enumerate}
\subsection{Aggiunta di persona al dataset per operazione Look-alike}
Se invece si vuole aggiungere persone nel dataset dei Look-alike serve fare alcuni passaggi aggiuntivi e il primo punto cambia per modalità online (via camera) e offline (img nel dataset)
\begin{enumerate}
    \item Aggiungere alla dir \texttt{data/similarity\_images/known\_people} l'immagine con nome quello della persona che si vuole aggiungere alla lista di persone che si volgiono confrontare con i Look-alike
    \item Aggiungere alla dir \texttt{data/similarity\_images/people} le immagini di persone "sconosciute" con cui si vuole fare un confronto (nota che questo passaggio è abbastanza computational intensive data la mole delle immagini. Consiglio di NON fare aggiungere e quindi ricreare gli embeddings della dir similarity\_images/people)
    \item Per creare gli embeddings dalle immagini in similarity\_images:\\
    \texttt{python -m src.extract\_embeddings}\\
    Nota che di base è commentata la parte di codice che fa l'embeddings delle similarity\_images/people per sicurezza (ultime righe del file .py)
    \item Si può far partire i main interessati al look-alike comparison con\\
    \texttt{python -m src.main\_look\_alike\_offline}   per confrontare immagini in \\ 
    similarity\_images/known\_people;\\
    \texttt{python -m src.main\_look\_alike\_online}    per confrontare direttamente attraverso la camera.
\end{enumerate}

\section{GUI app}
È possibile utilizzare un applicazione grafica per semplificare l'interazione con l'app. 
Il comando per far partire la gui:\\
\texttt{python -m src.gui\_app}
\subsection{Spiegazione GUI}
\begin{figure}[h]       % 'h' = here, posizionamento approssimativo
    \centering          % centra l'immagine
    \includegraphics[width=1\textwidth]{Immagini_Relazione/GUI_App.png}
    \caption{GUI App}
    \label{fig:guiapp} % per fare riferimenti all'immagine
\end{figure}
\begin{enumerate}
    \item \textbf{Import \& Process Images:} Consente di inserire le immagini in modo semplice. Basta specificare il nome della persona da riconoscere e selezionare l'immagine dal File System. Successivamente, premendo il pulsante ``Import \& Process'', vengono creati gli embeddings delle immagini. Questi embeddings vengono utilizzati nelle sezioni 3.a, 3.b, 3.c e 3.d.
    
    \item \textbf{Load Images to Classify:} Caricando le immagini in questa sezione, esse diventano disponibili nella sezione 3.b per la classificazione.
    
    \item \textbf{Run Main Script:} Quattro pulsanti consentono di avviare i diversi script principali dell'applicazione:
    \begin{enumerate}
        \item \textbf{Camera Recognition:} utilizzo della telecamera per la classificazione in tempo reale.
        \item \textbf{Image Recognition:} classificazione delle immagini caricate nella sezione 2.
        \item \textbf{Look-alike Offline:} ricerca di persone simili a quelle caricate.
        \item \textbf{Look-alike Online:} utilizzo della telecamera per la ricerca di persone simili a quelle caricate.
    \end{enumerate}
    
    \item \textbf{Utilities:} Strumenti aggiuntivi per la gestione degli embeddings e dei dataset.
    \begin{enumerate}
        \item \textbf{Re-extract All Embeddings:} Se viene caricata un'immagine al di fuori della GUI, questo pulsante verifica le directory modificate e crea gli embeddings mancanti.
        \item \textbf{Force Re-extract All:} genera gli embeddings per tutte le immagini senza controlli.
        \item \textbf{Show Dataset Info:} mostra la lista delle directory contenenti le immagini delle persone.
    \end{enumerate}
    
    \item \textbf{Console Output:} Visualizza messaggi e informazioni relative alle operazioni eseguite.
\end{enumerate}

\end{document}
